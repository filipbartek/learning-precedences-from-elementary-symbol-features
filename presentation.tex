\documentclass[notes]{beamer}

\documentclass[a4paper]{easychair}

\usepackage[utf8]{inputenc}
\usepackage[T1]{fontenc}
\usepackage[english]{babel}

\usepackage{amsmath}
\usepackage{amssymb}
\usepackage{todonotes}

\usepackage[numbers]{natbib}

\usepackage{glossaries}
\makeglossaries

\newacronym{atp}{ATP}{automated theorem prover}
\newacronym{fol}{FOL}{first order logic}

\newglossaryentry{problem}
{
	name=problem,
	description={a collection of zero or more premises and one or more conjectures}
}

\newglossaryentry{precedence}
{
	name=precedence,
	description={a total order on symbols in a given \gls{problem}}
}

\newglossaryentry{preference-value}
{
	name=preference value,
	description={a real number that expresses the expected contribution of a symbol being ordered before another symbol}
}

\newglossaryentry{preference-matrix}
{
	name=preference matrix,
	description={a square matrix populated with \glspl{preference-value} of all the pairs of symbols in a given \gls{problem}}
}

\newglossaryentry{symbol-order-matrix}
{
	name=symbol order matrix,
	description={a binary square matrix with value 1 for each pair of symbols ordered according to a given \gls{precedence}}
}


\DeclareMathOperator{\loss}{loss}
\DeclareMathOperator{\base_loss}{loss_{base}}
\DeclareMathOperator{\prover}{prover}
\DeclareMathOperator{\predicates}{\Sigma_P}
\DeclareMathOperator{\re}{\mathbb{R}}

\newcommand{\inv}[1]{#1^{-1}}


\title{Learning Precedences from Simple Symbol Features\thanks{Supported by the ERC Consolidator grant AI4REASON no. 649043 under the EU-H2020 programme and the Czech Science Foundataion project 20-06390Y.}}
\author{Filip B\'{a}rtek \and Martin Suda}
\institute{Czech Technical University in Prague, Czech Republic}

\usepackage[pdf]{graphviz}

\usetheme{Madrid}

% https://tex.stackexchange.com/a/35175/202639
\makeatletter
\setbeamertemplate{footline}
{
	\leavevmode%
	\hbox{%
		\begin{beamercolorbox}[wd=.25\paperwidth,ht=2.25ex,dp=1ex,center]{author in head/foot}%
			\usebeamerfont{author in head/foot}\insertshortauthor%~~\beamer@ifempty{\insertshortinstitute}{}{(\insertshortinstitute)}
		\end{beamercolorbox}%
		\begin{beamercolorbox}[wd=.50\paperwidth,ht=2.25ex,dp=1ex,center]{title in head/foot}%
			\usebeamerfont{title in head/foot}\insertshorttitle
		\end{beamercolorbox}%
		\begin{beamercolorbox}[wd=.25\paperwidth,ht=2.25ex,dp=1ex,right]{date in head/foot}%
			\usebeamerfont{date in head/foot}\insertshortdate{}\hspace*{2em}
			\insertframenumber{} / \inserttotalframenumber\hspace*{2ex} 
	\end{beamercolorbox}}%
	\vskip0pt%
}
\makeatother

\begin{document}

\begin{frame}
\titlepage
\end{frame}

\section{Introduction}

\begin{frame}{Context}
% TODO: Is there a better title?
\centering
\digraph[scale=0.3]{context}{
	node [shape=rectangle, fontsize=32];
	atp [label="Automated theorem proving for first-order logic"];
	refutation [label="Refutation-based"];
	kbo [label="Knuth-Bendix Ordering"];
	lpo [label="Lexicographic Path Ordering"];
	precedence [label="Symbol precedence", style=bold];
	"Automated reasoning" -> atp -> refutation -> "Superposition calculus" -> "Term simplification ordering" -> kbo -> precedence;
	atp -> "Saturation-based" -> "Superposition calculus";
	"Term simplification ordering" -> lpo -> precedence;
}
% Does superposition calculus really use paramodulation inference?
\end{frame}
\note{KBO is further parameterized by symbol weights. Vampire uses uniform weights so we ignore this parameter.}

\begin{frame}{Basic concepts}
\begin{table}
\begin{tabular}{lll}
	Concept & Notation & Example \\
	\hline
	First-order problem & $P = (\symbols, \clauses)$ & \\
	Symbols & $\symbols = (s_1, s_2, \ldots, s_n)$ & \\
	Symbol precedence & $\pi_P \sim \mathit{Perm}(\symbols)$ & $(0, 3, 1, 2)$ \\
	% Note that the set \pi_P is drawn from does not demonstrate notation consistently.
	Order matrix & $\OrderMatrix(\pi_P)$ & $\begin{bmatrix}
	0 & 1 & 1 & 1 \\
	0 & 0 & 1 & 0 \\
	0 & 0 & 0 & 0 \\
	0 & 1 & 1 & 0
	\end{bmatrix}$ \\
	Flattened matrix & $\flatten{\OrderMatrix(\pi_P)}$ & 0111001000000110 \\
	Abstract solving time & $\AbstractTime(P,\pi_P)$ & \\
	%Linear regression & & $\hat{y} =  \mathbf{x} \cdot \mathbf{w} + b$ \\
	Training problem set & $\ProblemsTrain$ & \\
\end{tabular}
\end{table}
\end{frame}
\note{Order matrix: The element in the $i$-th row and $j$-th column is 1 iff the permutation $\pi_P$ orders $s_i$ before $s_j$, that is $\inv{\pi_P}(i) < \inv{\pi_P}(j)$.
The element is 0 otherwise.}

\begin{frame}{Main research question}

\centering

How can we propose good symbol precedences

using simple symbol features?

\end{frame}

\section{Architecture}

\begin{frame}{Architecture overview}

Training:
\begin{enumerate}
	% TODO: Should I specify number of training problems and precedences here?
	\item For each of 1000 training problems in isolation:
	\begin{enumerate}
		\item Try to solve with 100 random precedences using Vampire
		% TODO: Should I mention Vampire here?
		\item Assign a cost to each of the precedences
		\item Assign a cost to each symbol pair
		\begin{enumerate}
			\item Fit a linear regressor of precedence cost
			\item Extract feature coefficients (feature $\sim$ symbol pair)
		\end{enumerate}
	\end{enumerate}
	\item Train a symbol pair cost regressor
	\begin{itemize}
		\item Represent each symbol by simple syntactic features
	\end{itemize}
\end{enumerate}

Proposing a precedence on a new problem:
\begin{enumerate}
	\item Predict the cost of every symbol pair
	\item Construct a precedence that minimizes the cumulative symbol pair cost
\end{enumerate}

\end{frame}

\begin{frame}{Precedence cost}
For each problem $P \in \ProblemsTrain$:
\begin{enumerate}
	\item For 100 uniformly random precedences $\pi_P$:

	Run Vampire and measure $\AbstractTime(P,\pi_P)$.
	\item Assign a penalty to precedences that fail to solve the problem (namely by timing out).
	% TODO: Consider leaving out to save time.
	\begin{itemize}
		\item Penalty: Maximum $\AbstractTime(P,\pi_P)$ over successful runs on $P$
	\end{itemize}
	\item Normalize the costs:
	% TODO: Consider leaving out to save time.
	\begin{enumerate}
		\item Log-scale
		\item Standardize (ensure mean 0 and standard deviation 1)
	\end{enumerate}
\end{enumerate}
Let $\CostStd(\pi_P)$ be the normalized cost of $\pi_P$.
\end{frame}
\note{
	Why not multiply the penalty by some constant? -- Multiplying by 2 and 10 yields a very similar performance to keeping the maximum.
	
	Why log-scale? -- Because the visual inspection shows that the distributions are usually approximately log-normal.
	
	How to standardize? -- Subtract mean and divide by standard deviation.
}

\begin{frame}{Symbol pair cost}
For each problem $P \in \ProblemsTrain$, train a linear model:

\begin{itemize}
	\item Input: $\OrderMatrix(\pi_P)$ -- order matrix of a precedence
	
	Example: $\pi_P = (0, 3, 1, 2),
	\OrderMatrix(\pi_P) =
	\begin{bmatrix}
	0 & 1 & 1 & 1 \\
	0 & 0 & 1 & 0 \\
	0 & 0 & 0 & 0 \\
	0 & 1 & 1 & 0
	\end{bmatrix}
	$
	\item Target: $\CostStd(\pi_P)$ -- normalized cost of $\pi_P$
	% TODO: Consider leaving out to save time.
	\item Regression method: Lasso (linear regression with $L_1$ penalty)
	% https://scikit-learn.org/stable/modules/generated/sklearn.linear_model.Lasso.html
	\item Trained model coefficients: $W_P$ -- preference matrix
	
	${W_P}_{i,j} \sim$ cost attributed to $\iverson{s_i\text{ is ordered before }s_j}$
	\item Prediction: $\CostProxy(\pi_P, W_P) = \flatten{\OrderMatrix(\pi_P)} \cdot \flatten{W_P}$
\end{itemize}
\end{frame}
\note{
	Lasso uses squared loss.
	
	We use cross-validation to set the regularization parameter of Lasso
	to ensure the model generalizes to unseen precedences.
	
	Design flaw:
	We should multiply the coefficients by number of unordered symbol pairs, that is $n \choose 2$,
	and predict by averaging instead of summation.
	Otherwise small problems dominate.
}

\begin{frame}{Symbol features}
6 simple features:
\begin{itemize}
	\item Arity
	\item Frequency (number of occurrences)
	\item Number of clauses that contain the symbol
	\item Is it in conjecture?
	\item Is it in a unit clause?
	\item Is it introduced (Tseitin or Skolem)?
\end{itemize}
Feature vector of symbol $s_i$: $\fv(s_i)$
\end{frame}
\note{
	Why these features? -- They are readily available in Vampire
	and they suffice as a basis for common precedence generation schemes such as invfreq.
	
	The last 3 features are indicator features.
}

\begin{frame}{Symbol pair cost model $M$}{Generalizing across problems}
\begin{itemize}
	\item Input: symbol pair feature vector $[\fv(s_i), \fv(s_j)]$
	\item Target: ${W_P}_{i,j}$ (cost attributed to $\iverson{s_i\text{ is ordered before }s_j}$)
	\item Prediction: $M([\fv(s_i), \fv(s_j)])$
	\item Learning method:
	\begin{itemize}
		\item Elastic-Net (linear regression with $L_1$ and $L_2$ penalties)
		\item Gradient Boosting Machine
	\end{itemize}
	\item Training sample distribution:
	\begin{enumerate}
		\item Sample problems from $\ProblemsTrain$ uniformly.
		\item Given a problem, weight symbol pairs by absolute target value.
	\end{enumerate}
\end{itemize}
\end{frame}
\note{Why symbol weighting? -- Experiments show that it works better.}

\begin{frame}{Proposing a good precedence}
Given a previously unseen problem $P$:
\begin{enumerate}
	\item Estimate a preference matrix $\widehat{W_P}$.
	
	$\widehat{W_P}_{i,j} = M([\fv(s_i), \fv(s_j)])$.
	\item Construct a precedence \(\widehat{\pi_P}\) that approximately minimizes
	$\CostProxy(\widehat{\pi_P}, \widehat{W_P}) = \flatten{\OrderMatrix(\widehat{\pi_P})} \cdot \flatten{\widehat{W_P}}$.
	\begin{itemize}
		\item NP-hard in general
		\item 2-approximation algorithm by \citet{Cohen2011}:
		
		% TODO: Consider leaving out to save time.
		Repeatedly append symbol with the smallest potential:
		
		$\potential(s_i) = \sum_{s_j \in \SymbolsAvail} \widehat{W_P}_{i, j} - \sum_{s_j \in \SymbolsAvail} \widehat{W_P}_{j, i}$
	\end{itemize}
\end{enumerate}
\end{frame}
\note{
	The approximation algorithm yields the optimum solution in case we use a linear model
	of symbol pair cost.
	This has not been proven in the paper.
}

\section{Experiments}

\begin{frame}{Experimental evaluation}

\begin{itemize}
	\item Learning only predicate precedences
	\item 5 evaluation iterations, each using:
	\begin{itemize}
		\item 1000 training problems
		\item 1000 test problems
	\end{itemize}
\end{itemize}

\begin{table}
\begin{tabular}{l|rr}
	Case & \multicolumn{2}{c}{Successes out of 1000} \\
	& Mean & Std \\
	\hline
	Best of 10 random & 501.8 & 13.85 \\
	\texttt{invfreq} & 475.8 & 13.83 \\
	Elastic-Net & 471.6 & 10.21 \\
	Gradient Boosting & 464.8 & 13.78 \\
	Elastic-Net without weighting & 461.2 & 8.11 \\
	Random & 450.4 & 10.25 \\
\end{tabular}
\end{table}
\end{frame}

\begin{frame}{Elastic-Net feature coefficients}{of symbol pairs}
\begin{table}
\begin{tabular}{c|rrr|rrr}
	Split & \multicolumn{3}{c}{Left symbol} & \multicolumn{3}{c}{Right symbol} \\
	& Arity & Freq. & Unit freq. & Arity & Freq. & Unit freq. \\
	\hline
	0 &     &$-0.01$&     &$-0.98$&$ 0.01$&      \\
	1 &     &$-0.48$&     &     &$ 0.08$&$ 0.44 $\\
	2 &     &     &$-0.64$&     &$ 0.36$&      \\
	3 &$ 0.88$&$-0.03$& 0.01&     &$-0.03$&$ 0.05 $\\
	4 &     &$-0.62$&     &     &$ 0.30$&$ 0.07 $\\
	\hline
	\(\ProblemsTrain\) &     &     &$-0.57$&     &$ 0.43$&      \\
\end{tabular}
\end{table}

The other 3 features have all zero coefficients.
\end{frame}

\begin{frame}{Elastic-Net feature coefficients}{of individual symbols}
% TODO: Is it ok to include this slide, despite the fact that the table is not in the article?
% Perhaps make it a bonus slide?

\begin{table}
\begin{tabular}{c|rrr}
	Training set & Arity & Frequency & Unit frequency \\
	\hline
	0 & $-.98$ & $ .01$ & $-.01$ \\
	1 &        & $ .56$ & $ .44$ \\
	2 &        & $ .36$ & $ .64$ \\
	3 & $-.88$ &        & $ .04$ \\
	4 &        & $ .93$ & $ .07$ \\
	\hline
	\(\ProblemsTrain\) &        & $ .43$ & $ .57$ \\
\end{tabular}
\end{table}

Symbol order: descending by predicted value

\begin{itemize}
	\item Sets 1, 2, 4, $\ProblemsTrain$:
	\begin{itemize}
		\item Descending by frequency: low frequency $\sim$ early inference
		\item Similar to \texttt{invfreq} and \texttt{vampire \--\--sp frequency}
	\end{itemize}
	\item Sets 0, 3:
	\begin{itemize}
		\item Ascending by arity: high arity $\sim$ early inference
		\item Similar to \texttt{vampire \--\--sp arity}
	\end{itemize}
\end{itemize}
% https://docs.google.com/spreadsheets/d/1enrsKzIjNpEbXaZ8JtKBUe29N1C-S2Yf2jDAPUsizDA

\end{frame}

\section{Conclusion}

\begin{frame}{Summary}

Training:
\begin{enumerate}
	\item For each training problem in isolation:
	\begin{enumerate}
		\item Run Vampire on 100 random precedences.
		\item Assign a cost to each symbol pair:
		Fit a linear regressor of precedence cost and extract feature coefficients.
	\end{enumerate}
	\item Train a symbol pair cost regressor.
	
	Represent symbols using simple syntactic features.
\end{enumerate}

Proposing a precedence on a new problem:
\begin{enumerate}
	\item Predict the cost of each symbol pair.
	\item Minimize the sum of costs of ordered symbol pairs.
\end{enumerate}

Experimental evaluation:
\begin{itemize}
	\item We learned to order predicate symbols by frequency.
\end{itemize}

Future:
\begin{itemize}
	\item Stabilize training
	\item Learn to propose function symbol precedences
	\item Richer symbol representation using graph neural networks
\end{itemize}

\end{frame}

\section{Bonus slides}

\begin{frame}[noframenumbering]
\sectionpage
\end{frame}

\begin{frame}[noframenumbering]{References}
\bibliographystyle{plainnat}
\bibliography{main}
\end{frame}

\begin{frame}[noframenumbering]{Experimental setup}
\begin{itemize}
	\item Only predicate precedences are learned.
	Function symbols are ordered by \texttt{invfreq}.
	\item Problems from TPTP \cite{Sut17} -- FOF and CNF
	\begin{itemize}
		\item $\ProblemsTrain$ (8217 problems):
		at most 200 predicate symbols,
		at least 1 out of 24 random predicate precedences yield success
		\item $\ProblemsTest$ (15751 problems):
		at most 1024 predicate symbols
	\end{itemize}
	\item 5 evaluation iterations (splits): 1000 training problems and 1000 test problems
	\item 100 precedences per training problem
	\item Vampire: time limit: 10 seconds
	\item $10^6$ symbol pair samples to train $M$
\end{itemize}
\end{frame}

\end{document}
